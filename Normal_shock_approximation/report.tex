\documentclass{article}
%Preamble
\usepackage{float}
\usepackage{color}
\usepackage{listings}
\usepackage{longtable}
\usepackage{amsmath,amssymb}
\usepackage{graphics}
\usepackage{graphicx}

\title{AE 622 -  Computing of high speed flows\\ Assignment 2: Report \\ Numerical viscocity and Shock capturing}
\author{Vinod Kumar Metla - 130010048\\Aditi Taneja - 13D100026}
\date{}

\begin{document}
\pagenumbering{arabic}
\maketitle
\newpage
\section*{Introduction}

Normal shock tube is simulated in octave with initial conditions as - 
\newline
    $\begin{bmatrix}
    \rho_l \\ 
    u_l \\
    p_l \\
    T_l
    \end{bmatrix} = \begin{bmatrix}
    1.0\\ 
    10.0 \\
    0.7148 \\
    1.0
    \end{bmatrix} $
 , 
$\begin{bmatrix}
    \rho_r \\ 
    u_r \\
    p_r \\
    T_r
    \end{bmatrix} = \begin{bmatrix}
    5.7143\\ 
    1.75 \\
    83.273 \\
    20.387
    \end{bmatrix} $
\\
Simulations are carried out by varying alpha, Grid size, and CFL number. Their effects on $\rho$, u, p, T are shown in the graphs with corresponding L2 errors plots in u. All values are non-dimensional.
\newline
Exact solution is plotted below for the same initial conditions.
\section*{Dependencies}
\begin{enumerate}
\item Pdflatex 
\item Octave
\item Numpy
\item Matplotlib

\end{enumerate}
\newpage

\section*{Exact Solution}
The exact solution for a normal shock for the initial conditions given above:

\begin{figure}[H]   \label{figure}
\includegraphics[width=15cm]{exact.png}
\label{figure:}
\end{figure}
\newpage
\section*{Varying $\alpha$}

Following Plots were obtained for $\alpha$ = $[0.9 , 0.6 ,0.4, 0.3, 0.2, 0.15 ]$
\newline
CFL number was kept constant at 0.2
\newline
Grid Size = 100
\newline
Mach Number = 10.0 (Hypersonic)

\begin{figure}[H]   \label{figure}
\includegraphics[width=15cm]{alpha.png}
\label{figure:}
\end{figure}

\begin{figure}[H]   \label{figure}
\includegraphics[width=15cm]{err_alpha.png}
\caption{ L2 error in u with varying $\alpha$}
\label{figure:}
\end{figure}

\begin{description}
\item[]As $\alpha$ is decreased, approximate solution becomes closer to the exact solution, L2 error in u decreases.
\item[]As $\alpha$ is decreased, the effect of diffusion decreases and dispersion increases. This happens because the second order derivative term in U contributes to diffusion (even order derivatives), and the first order derivative contributes to dispersion ( odd order derivatives).So, when alpha is decreased, the contribution of dissipative/diffusion term decreases and in turn, effect of dispersion term increases.
\item[]Minimum alpha for given CFL, Mach number and Grid size lies between 0.145- 0.15.

\end{description}
\newpage

\section*{Varying Grid size(n)}

Following Plots were obtained for n = $[100 , 200 , 400, 600, 800, 1000]$
\newline
CFL number was kept constant at 0.2
\newline
$\alpha$ = 0.5
\newline
Mach Number = 10.0 (Hypersonic)

\begin{figure}[H]   \label{figure}
\includegraphics[width=15cm]{grid.png}
\label{figure:}
\end{figure}

\begin{figure}[H]   \label{figure}
\includegraphics[width=15cm]{err_grid.png}
\caption{ L2 error in u for varying grid size}
\label{figure:}
\end{figure}



\begin{description}
\item[] As grid size is increased, approximate solution obtained for normal shock tends more towards exact solution. Dissipation as well as dispersion effects are less significant.
\item[] As grid size is increased, cost of computation increases. Thus, there is a trade off between accuracy and cost of comptation. Optimum value of Grid size can be obtained from the L2 error plot for u. 
\item[] From our analysis, optimum value of grid size should be close to 600 because error does not decrease much as the grid size is increased from 600 to 800.
\end{description}
\newpage

\section*{Varying CFL number}

Following Plots were obtained for CFL = $[0.9, 0.7, 0.5, 0.3, 0.2,0.1]$
\newline
Grid Size = 100
\newline
$\alpha$ = 0.6
\newline
Mach Number = 10.0 (Hypersonic)

\begin{figure}[H]   \label{figure}
\includegraphics[width=15cm]{cfl.png}
\label{figure:}
\end{figure}

\begin{figure}[H]   \label{figure}
\includegraphics[width=15cm]{err_cfl.png}
\caption{ L2 error in u with varying CFL}
\label{figure:}
\end{figure}

\begin{description}
\item[]As CFL number is decreased, relative error decreases and therefore, the approximate solution tends more towards exact solution but, number of computations increase with decreasing $\delta t$ if final time remains. However, here number of time steps are kept constant = 10000.
\item[] With decreasing CFL number, diffusion effect also decreases.
\item[] As CFL is increased, minimum value of alpha for which the code does not diverge increases.
\end{description}

\newpage

\section*{Additional Questions}
\begin{enumerate}
  \item \textbf{There are no boundary conditions specified in the code. Why?}
\\
  Boundary conditions are not specified because values for $\rho$, u, p, T were specified in the initial conditions and do not change at the boundaries. Computations are only carried out at grid points from 2 to n-1. Hence, no need to specify boundary conditions.
  \item \textbf{What happens if you have a non-uniform grid?}
\\
  Fine grid near/around the shock and coarser grid away from the shock will be beneficial, since it will give better solutions near the shock and minimize redundant calculations.
  \item \textbf{Are there any effect of initial conditions?}
\\ 
 Yes, if initial conditions are changed, Mach number upstream the shock will change, due to which all properties ( u, $\rho$, p, T) upstream and downstream of the shock will change. Thus, different initial conditions will lead to different shock strengths.
\end{enumerate}

\end{document}
